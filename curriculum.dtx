% \iffalse meta-comment
%<*internal>
\begingroup
\input docstrip
\preamble
----------------------------------------

Francesco Contini, La classe curriculum,
doc. lang. italian, Cagliari 2025.

Copyright (C) 2025 Francesco Contini

This work consists of the README.md, the
curriculum.dtx and the derived:
- curriculum.cls,
- curriculum.xmpdata,
- curriculum.bib,
- curriculum.pdf.
The .pdf derived refers to the preamble,
from the .dtx source, and reproduces its
its text.
The work may be distributed and modified
under the conditions of the ``LaTeX Proj
ect Public License,'' either version 1.3
of the the license or at your option any
later version. For the latest version of
this license see the url below.
The product has got the LPPL maintenance
status `maintained'.
The maintainer of this work is Francesco
Contini. Email to francescocontini98[at]
gmail[dot]com to claim and feed back, or
to get a clarification.

 https://www.latex-project.org/lppl.txt

----------------------------------------
\endpreamble
\postamble
----------------------------------------

Francesco Contini, The curriculum class,
Cagliari 2025.

Copyright (C) 2025 Francesco Contini

Available under the LaTeX Project Public
License,  which is published on https://
www.latex-project.org/lppl.txt.

---------------------------------------
\endpostamble
\keepsilent
\askforoverwritefalse

\Msg{*** Generating the style file ***}
\generate{
    \file{curriculum.cls}{\from{curriculum.dtx}{class}}
    \file{curriculum.xmpdata}{\from{curriculum.dtx}{data}}
    \file{curriculum.bib}{\from{curriculum.dtx}{bib}}}

\obeyspaces
\Msg{                                                  }
\Msg{                                                  }
\Msg{**************************************************}
\Msg{*                                                *}
\Msg{* To finish the installation, move the following *}
\Msg{* files into a directory searched by TeX.        *}
\Msg{*                                                *}
\Msg{* curriculum.cls                                 *}
\Msg{* curriculum.pdf                                 *}
\Msg{* README.md                                      *}
\Msg{*                                                *}
\Msg{* To produce the documentation, run the dtx file *}
\Msg{* from a shell through LaTeX. See the README for *}
\Msg{* for more details.                              *}
\Msg{**************************************************}
\Msg{                                                  }
\Msg{                                                  }
\ifx\directlua\undefined
\Msg{**************************************************}
\Msg{* LuaLaTeX engine not detected.                  *}
\Msg{* Executing \string\endbatchfile.                       *}
\Msg{**************************************************}
\Msg{                                                  }
\Msg{                                                  }
\endbatchfile
\else
\Msg{**************************************************}
\Msg{* LuaLaTeX engine detected.                      *}
\Msg{* Skipping \string\endbatchfile.                        *}
\Msg{**************************************************}
\Msg{                                                  }
\Msg{                                                  }
\fi
\endgroup
%</internal>
% \fi
%
%
%
% \iffalse
%<*driver>
\ProvidesFile{curriculum.dtx}
%</driver>
%
%
%
%<class>\NeedsTeXFormat{LaTeX2e}
%<class>\ProvidesClass{curriculum}
%<*class>
    [2025/08/19 v.1.0 Curriculum]

%</class>
%
%
%
%<*driver>
\documentclass[12pt]{ltxdoc}
% --------------------------------------------------------
% TYPEFONT SETUP
% --------------------------------------------------------
\usepackage[osf]{libertinus}
% --------------------------------------------------------
% LANGUAGE SETUP
% --------------------------------------------------------
\usepackage{polyglossia}
    \setmainlanguage{italian}
    \setotherlanguage[variant=british]{english}
    \setotherlanguage{french}
    \SetLanguageKeys{italian}{indentfirst=false}
    \usepackage{fmtcount}
    \RequirePackage{itnumpar}
    \FCloadlang{italian}
%--------------------------------------------------------
% GRAPHICS AND CODES
%--------------------------------------------------------
\usepackage{xcolor}
    \definecolor{cvgray}{gray}{0.25}
    \definecolor{cvblue}{rgb}{0.25,0.25,0.50}
    \definecolor{cvdarkblue}{rgb}{0.0,0.0,0.25}
    \definecolor{cvgreen}{rgb}{0.25,0.50,0.25}
    \definecolor{cvred}{rgb}{0.75,0.0,0.0}
    \definecolor{cvyellow}{rgb}{0.95,0.65,0.25}
    \definecolor{lightgray}{gray}{0.975}
    \definecolor{lightblue}{rgb}{0.95,0.95,1.00}
\usepackage{listings}
    \lstdefinestyle{fromthesource},
        keywordstyle=\color{cvred},
        numberstyle=\tiny\color{cvred},
        stringstyle=\color{cvred},
        breakatwhitespace=true,
        breaklines=true,
        keepspaces=false,
        aboveskip=\bigskipamount,
        belowskip=\bigskipamount,
        captionpos=b,
        belowcaptionskip=0pt,
        numbers=none,
        showtabs=true,
        showspaces=true,
        showstringspaces=true,
        numberbychapter=false,
        extendedchars=false}
    \lstdefinestyle{inputexample},
        keywordstyle=\color{cvyellow},
        numberstyle=\tiny\color{cvred},
        stringstyle=\color{cvdarkblue},
        breakatwhitespace=true,
        breaklines=true,
        keepspaces=false,
        captionpos=b,
        belowcaptionskip=0pt,
        numbers=left,
        numbersep=10pt,
        stepnumber=5,
        numberfirstline=false,
        firstnumber=1,
        numberbychapter=false,
        aboveskip=0.5pc,belowskip=0.5pc}
\renewcommand\lstlistlistingname{Lista dei codici illustrativi}
\renewcommand{\lstlistingname}{Codice}
% --------------------------------------------------------
% OTHER PACKAGES
% --------------------------------------------------------
\usepackage[style=accursius,alldates=short]{biblatex}
    \addbibresource{curriculum.bib}
\usepackage{caption}
    \captionsetup{font=small,textformat=period}
\usepackage[style=italian]{csquotes}
\usepackage[ddmmyyyy]{datetime}
    \newcommand{\defaulttoday}
        {\twodigit{\day}.\twodigit{\month}.\twodigit{\year}}
\usepackage{fontawesome}
\usepackage{graphicx}
\usepackage{lettrine}
\usepackage{paralist}
    \usepackage{enumitem}
        \setlist[itemize]{label={-},noitemsep,topsep=0pt}
        \setlist[enumerate]{noitemsep,topsep=0pt}
        \setlist[description]{noitemsep,topsep=0pt}
% --------------------------------------------------------
% HYPERREF SETUP
% --------------------------------------------------------
\usepackage{luatex85}
\usepackage[a-1a]{pdfx}
\usepackage[pdfa]{hyperref}
% --------------------------------------------------------
% FINAL COMMANDS
% --------------------------------------------------------
\newrobustcmd*{\emphitem}[1]{\textcolor{rdarkgray}{\emph{#1}}}
\raggedbottom
\newcommand{\omissis}{[\ldots\unkern]}
\newcommand{\changecapital}[1]{[#1\unkern]}
\EnableCrossrefs
\RecordChanges
\begin{document}
    \DocInput{curriculum.dtx}
    \PrintChanges
\end{document}
%
%</driver>
% \fi
%
% \CheckSum{953}
%
% \CharacterTable
%  {Upper-case    \A\B\C\D\E\F\G\H\I\J\K\L\M\N\O\P\Q\R\S\T\U\V\W\X\Y\Z
%   Lower-case    \a\b\c\d\e\f\g\h\i\j\k\l\m\n\o\p\q\r\s\t\u\v\w\x\y\z
%   Digits        \0\1\2\3\4\5\6\7\8\9
%   Exclamation   \!     Double quote  \"     Hash (number) \#
%   Dollar        \$     Percent       \%     Ampersand     \&
%   Acute accent  \'     Left paren    \(     Right paren   \)
%   Asterisk      \*     Plus          \+     Comma         \,
%   Minus         \-     Point         \.     Solidus       \/
%   Colon         \:     Semicolon     \;     Less than     \<
%   Equals        \=     Greater than  \>     Question mark \?
%   Commercial at \@     Left bracket  \[     Backslash     \\
%   Right bracket \]     Circumflex    \^     Underscore    \_
%   Grave accent  \`     Left brace    \{     Vertical bar  \|
%   Right brace   \}     Tilde         \~}
%
% \changes{v. 1.0}{2025/08/19}{Versione iniziale}
%
% \GetFileInfo{curriculum.cls}
% \DoNotIndex{\newcommand,\newenvironment}
%
% \author{Francesco Contini}
% \title{La classe \textsf{curriculum}}
% \date{Cagliari, \defaulttoday}
% \maketitle
% \tableofcontents
% \clearpage
%
% \section*{The license and the \texttt
%        {.dtx} source file}
% \iffalse
%<*example>
% \fi
\begin{lstlisting}[style=fromthesource,
    nolol,showspaces=false]
Francesco Contini, La classe curriculum,
doc. lang. italian, Cagliari 2025.

Copyright (C) 2025 Francesco Contini

This work consists of the README.md, the
curriculum.dtx and the derived:
- curriculum.cls,
- curriculum.xmpdata,
- curriculum.bib,
- curriculum.pdf.
The .pdf derived refers to the preamble,
from the .dtx source, and reproduces its
its text.
The work may be distributed and modified
under the conditions of the ``LaTeX Proj
ect Public License,'' either version 1.3
of the the license or at your option any
later version. For the latest version of
this license see the url below.
The product has got the LPPL maintenance
status `maintained'.
The maintainer of this work is Francesco
Contini. Email to francescocontini98[at]
gmail[dot]com to claim and feed back, or
to get a clarification.

 https://www.latex-project.org/lppl.txt
\end{lstlisting}
% \iffalse
%</example>
% \fi
%
% \clearpage
% \author{Francesco Contini}
% \title{La classe \textsf{curriculum}}
% \date{Cagliari, \defaulttoday}
% \maketitle
% \section{Istruzioni per l'uso}
% \subsection{Motivi d'una nuova
%    classe di \textit{cv}}
% \lettrine{L}{a scrittura} d'un \textit
% {curriculum vitae et studiorum} segue,
% di volta in volta, regole diverse.  Il
% contenuto e la stessa formattazione, %
% infatti, dipendono dal destinatario %%
% del \textit{cv}.
%
% Esistono numerosi \textit{softwares} %
% per scrivere un \textit{cv}.  Basta un
% un comune \textit{word processor} allo
% scopo. Sono molteplici gli applicativi
% in rete. Tra questi, \textit{europass}
% è lo strumento ufficiale per l'omonimo
% \textit{standard} e trova,  in \LaTeX,
% un'imitazione affidabile nel lavoro di
% \citeauthor{mazzamuto2020}.\footnote%%
% {Vedi \cite{mazzamuto2020}.}
%
% In generale, su \LaTeX\ esistono già %
% lavori che offrono, in sostanza, dei %
% modelli  per  un \textit{cv}.\footnote
% {\`{E} decisamente valido il lavoro di
% \cite{danaux2024}.  Per altri modelli,
% \cite{prasad2023}; \cite{scrivena2018};
% \cite{ghersi2012}; \cite{verna2010}; %
% \cite{maier2001}.}
%
% \`{E} per questo che ritengo di non %%
% dover depositare questa classe sul %%%
% \textsc{ctan}  e di limitarmi, invece,
% a caricarla su \citelist{contini2025}%
% {organization}.\footnote{Visita \cite%
% {contini2025}.}  In effetti, questa %%
% classe, \textsf{curriculum}, ha più %%
% ragioni personali —~quelle di chi è %%
% alla ricerca d'un \textit{cv} fatto %%
% bene~— che di condivisione. Il momento
% di diffusione è più un'opportunità che
% viene a valle, insomma.
%
% Questa classe persegue l'obiettivo %%%
% della semplicità nella scrittura del %
% \textit{cv}.  È la semplicità di chi %
% non necessita di funzioni particolari,
% ma solo di dare un resoconto assai %%%
% sommario, del percorso formativo e del
% \enquote{\textit{cursus honorum}.}  La
% sobrietà di questa classe è quasi %%%%
% rigida, perché trova temperamento solo
% nell'elemento decorativo dei colori, %
% che, di \textit{default}, marcano le %
% intestazioni e attenuano la lettura %%
% del testo.
%
% Certo, un \textit{cv} non è il luogo %
% dove fare sfoggio concreto di capacità
% editoriali e tipografiche: serve per %
% presentare se stessi —~semmai, cioè, %
% per indicare queste capacità~—.  Però,
% è altrettanto certo che funzioni più %
% elaborate necessitano dell'impiego %%%
% d'altre classi \LaTeX.
%
% \subsection{I dati personali}
% La classe riproduce i dati personali %
% nell'intestazione del \textit{cv} e, %
% in parte,  nelle testatine e nei piedi
% delle eventuali pagine successive alla
% prima.  Il resto della pagina, secondo
% questo schema, è il luogo per le altre
% informazioni:  quelle che, più che %%%
% identificare la persona, servono a %%%
% distinguerla dalla generalità sotto il
% profilo strettamente professionale.
%
% \DescribeMacro{\maketitle} Dunque, i %
% dati personali sono quelli che servono
% a identificare la persona fisica. Il %
% luogo per la loro riproduzione è %%%%%
% l'intestazione del documento.  Proprio
% per questo, la classe ridefinisce %%%%
% \texttt{\textbackslash maketitle}.
%
% I dati personali obbligatori sono il %
% nome e il cognome,  i cui comandi sono
% \texttt{\textbackslash name} e \texttt
% {\textbackslash surname}.  Tutti gli %
% altri dati personali sono facoltativi.
% Sono tutti comandi con argomento che è
% obbligatorio.
%
% Di seguito, il riepilogo di tutti i %%
% dati personali.
% \begin{description}[%
%         font=\normalfont\ttfamily \textbackslash]
%     \item[name] il nome, obbligatorio;
%     \item[surname] il cognome, anche %
%         questo obbligatorio;
%     \item[position] il mestiere o il %
%         titolo;
%     \item[birthday] data di nascita; %
%     \item[gender] il genere, con testo
%         vincolato, associato all'icona
%         binaria, come segue:
%         \begin{inparaenum}[\slshape a.]
%             \item \texttt{F},
%             \item \texttt{f},
%             \item \texttt{Female},
%             \item \texttt{female},
%             \item \texttt{Woman},
%             \item \texttt{woman},
%             \item \texttt{Femmina},
%             \item \texttt{femmina},
%             \item \texttt{Donna},
%             \item \texttt{donna},
%             \item \texttt{M},
%             \item \texttt{m},
%             \item \texttt{Male},
%             \item \texttt{male},
%             \item \texttt{Man},
%             \item \texttt{man},
%             \item \texttt{Maschio},
%             \item \texttt{maschio},
%             \item \texttt{Uomo} e
%             \item \texttt{uomo},
%             \item ma anche un'altra %%
%                 stringa va bene, per %
%                 quanto privo d'icona %
%                 associata;
%         \end{inparaenum}
%     \item[addresslineone] la prima %%%
%                 riga dell'indirizzo; %
%     \item[addresslinetwo] la seconda %
%                 riga dell'indirizzo; %
%     \item[email] indirizzo di posta %%
%                 elettronica ordinaria;
%     \item[pec] indirizzo \textsc{pec};
%     \item[phone] numero di telefono; %
%     \item[fax] numero di \textit{fax};
%     \item[linkedin] l'identificativo %
%                 del profilo \textit%%%
%                 {linkedin} e
%     \item[github] l'identificativo del
%                 profilo \textit{GitHub}.
% \end{description}
% Pertanto, l'omissione di questi dati %
% non comporta errori nella compilazione
% del \textit{cv}.
%
% \DescribeMacro{\portrait}
% A scelta, è possibile inserire anche %
% una foto,  con  \texttt{\textbackslash
% portrait}, inserendo in argomento il %
% nome del \textit{file}, estensione %%%
% compresa.
%
% \iffalse
%<*example>
% \fi
\begin{lstlisting}[caption={Esempio di
    dati personali in \textit{cv}},
    style=inputexample,showspaces=false,
    label={code:exampledatas}]
\documentclass{curriculum}
    \name{Nome}
    \surname{Cognome}
    \position{Laureata}
    \birthday{01.01.2000}
    \gender{donna}

    \addresslineone{via delle Lumache, 15}
    \addresslinetwo{00100 -- Roma, Italia}
    \email{username@domain.com}
    \pec{username@pecdomain.com}
    \phone{+39 06 0000 0000}
    \fax{+39 06 0000 0000}
    \mobile{+39 300 000 0000}
    \github{username}
    \linkedin{username}

    \portrait{mishlanova-unsplash.jpg}

\begin{document}
    \maketitle
\end{document}
\end{lstlisting}
% \iffalse
%</example>
% \fi
%
% Con tutti questi dati, è già possibile
% redigere l'intestazione del documento,
% comprensiva pure di foto.\footnote%%%%
% {La foto profila usata negli esempi  è
% tratta, in \textit{opensource}, da %%%
% \textsf{\citelist{mishlanova2021}{organization}}.
% Autrice è \citeauthor{mishlanova2021}.
% Vedi \cite{mishlanova2021}.}
%
% \begin{figure}
%     \centering
%     \includegraphics[width=\textwidth]
%         {maketitle.pdf}
%     \label{fig:exampledatas}
%     \caption{Anteprima del codice \ref
%         {code:exampledatas}}
% \end{figure}
%
% Non rientrano tra i dati personali, ma
% trovano un \textit{output} nel comando
% \texttt{\textbackslash maketitle}, %%%
% l'informazione del luogo in cui l'utente
% sta redigendo il \textit{cv} e una %%%
% stringa introduttiva di data e luogo %
% del \textit{cv} stesso.  La classi, %%
% quindi, mette a disposizione ulteriori
% due comandi, che sono facoltativi, con
% un argomento obbligatorio.
%
% \DescribeMacro{\place}  Col comando %%
% \textsf{place} l'utente indica da dove
% sta compilando il suo \textit{cv}.  La
% classe stampa sempre questo dato, se %
% presente, sia in intestazione, sempre,
% sia nello spazio dedicato alla %%%%%%%
% sottoscrizione del \textit{cv}.  Il %%
% dato precede sempre la data di %%%%%%%
% compilazione del documento.
%
% \LaTeX\ calcola automaticamente la %%%
% data di compilazione, che stampa in %%
% ogni caso.
%
% \DescribeMacro{\beforedate}
% La classe offre anche un comando per %
% scrivere una stringa,  in intestazione
% al \textit{cv}, a precere la data e, %
% se data,  la località di compilazione.
% Per eccellenza,  l'argomento di questo
% di comando si presta a specificare %%%
% che la data si riferisce all'ultimo %%
% aggiornamento del documento. La classe
% stampa questo campo in tondo e,  senza
% segni di punteggiatura,  lo fa seguire
% dallo spazio e dal luogo, se dato, o %
% direttamente dalla data di compilazione.
%
% \subsection{Impostazioni di riservatezza}
% Di base, la classe stampa tutti i dati
% personali che l'utente abbia dato.
%
% Come soluzione alla volontà di non %%%
% stampare un certo dato, l'utente può %
% semplicemente ricorrere all'omissione,
% cancellando il comando, così come %%%%
% nascondendolo col segno di commento %%
% \%.
%
% Infatti, a parte \textsf{name} e \textsf
% {surname}, tutti gli altri dati %%%%%%
% personali sono facoltativi.
%
% Tuttavia, è possibile, comunque,  dare
% anche tutti i dati personali, ma dare,
% al contempo, indicazione di non %%%%%%
% stamparli.  \`{E} un meccanismo che %%
% viene in contro, soprattutto, a chi %%
% scrive più \textit{cv} destinati a più
% destinatari, o anche alla pubblicazione.
% In un'eventualità del genere, forse, è
% più comodo disporre d'un \textit{input}
% completo di dati, di cui fare copia e,
% poi, da modificare inserendo l'opzione
% per la riservatezza, anzi che cancellando
% i dati.
%
% A questo scopo, la classe definisce %%
% varie opzioni, alcune cumulabili e %%%
% altre che s'assorbono,  per nascondere
% uno o più dati personali.
%
% \DescribeMacro{public} L'opzione %%%%%
% \texttt{public} oscura tutti i dati %%
% utili a rintracciare fisicamente o %%%
% digitalmente l'intestatario e cioè %%%
% l'indirizzo e i numero di telefono e %
% \textit{fax}, l'indirizzo \textsc{pec}
% e, infine, il numero di cellulare.  Si
% presta, insomma, molto bene ai \textit
% {cv} destinati alla pubblicazione, %%%
% specie in adempimento agli obblighi di
% trasparenza —~in particolare, gravanti
% sulle pubbliche amministrazioni~—.
%
% \DescribeMacro{privacy} L'opzione %%%%
% \texttt{privacy} attiva, anzitutto, %%
% l'opzione \texttt{public}. In aggiunta
% nasconde il compleanno e il genere.
%
% \DescribeMacro{letterhead} L'opzione %
% \texttt{letterhead} salta data di %%%%
% nascita e genere e, soprattutto, non %
% stampa la fotografia, a prescindere %%
% che sia caricata o meno.  Questa si %%
% presta, soprattutto, per documenti che
% vogliono, comunque, basarsi sulla %%%%
% classe \textsf{curriculum}, ma che non
% vogliono essere un \textit{cv}, né una
% lettera introduttiva o motivazionale %
% connessa a un \textit{cv}.
%
% Ulteriori due opzioni sono assorbite %
% dalle precedenti,  pur niente vietando
% il loro autonomo impiego.
%
% \DescribeMacro{nobirthdaynogender} Una
% è quella per sopprimere il genere e le
% data di nascita.
%
% \DescribeMacro{noaddress}  L'altra è %
% per saltare l'indirizzo fisico e %%%%%
% quello \textsc{pec}, oltre ai numeri %
% di telefono e \textit{fax}.
%
% \subsection{Decorazione del \textit{cv}}
% \DescribeMacro{blackandwhite}
% Il \textit{cv} è, di base, decorato %%
% col colore verde e col colore blu. %%%
% In particolare, il testo corrente è in
% una tonalità scura di blu, mentre i %%
% titoli delle sezioni e il nome in %%%%
% intestazione sono in una tonalità scura
% di verde.
%
% Tuttavia, l'utente può specificare, %%
% nel comando   \texttt  {\textbackslash
% document class}, l'opzione \texttt  %%
% {blackandwhite}, affinché il \textit%%
% {cv} sia, appunto, in bianco e nero.
%
% \subsection{Testatine del documento}
% Il \textit{cv} non ha, di base, alcuna
% testatina o piedino, a meno che non %%
% sia lungo due o più pagine.
%
% In quest'eventualità, le sole pagine %
% seconda e, eventualmente, successive %
% hanno una testatina e un piedino.
%
% La testatina indica il nome completo e
% la posizione, se data, dell'utente.
%
% Il piedino indica il numero di pagina,
% nonché il numero totale delle paginae,
% a destra.  Al centro, invece, il %%%%%
% piedino indica alcuni dati personali,
% compatibilmente con le impostazioni di
% di riservatezza che l'utente abbia %%%
% dato.
%
% \subsection{Il corpo del \textit{cv} e
%     le sue sezioni}
% Finita l'intestazione del \textit{cv},
% ne segue il corpo.  Come da buona %%%%
% prassi,  un buon \textit{cv} si divide
% in sezioni, tra le quali le diverse %%
% informazioni si ripartiscono secondo %
% un certo ordine.
%
% Le due sezioni importanti sono quelle
% relative alle attività lavorative in %
% corso e passate e all'esperienza  %%%%
% formativa, cui accedono sezioni meno %
% rigide e vincolate, quali quelle che %
% riepilogano le competenze, tra cui le
% c.d\@. \textit{soft skills}.
%
% \DescribeMacro{\section} La classe
% \textsf{curriculum} conferma il tipico
% sezionamento \textit{standard} di %%%%
% \LaTeX.  Sotto il profilo visivo, la %
% classe \textsf{curriculum} evidenzia %
% il primo livello di sezionamento con %
% un titolo indentato e in maiuscoletto.
% L'indentazione è decorata con una %%%%
% linea spessa 0.4pt, a rimarcare che %%
% inizia una sezione.  La dimensione del
% testo è \texttt{large}.
%
% \DescribeMacro{\subsection} C'è %%
% anche il secondo livello di sezioni, %
% \texttt{\textbackslash subsection}. %%
% Non è indentato, distinguendosi solo %
% per la dimensione \texttt{large} e per
% essere scritto in \textit{sans serif}.
%
% In astratto, sono possibili altri %%%%
% livelli di sezionamento,  ma la classe
% non provvede a personalizzarli, perché
% ritiene in partenza che non siano %%%%
% necessari a un \textit{cv} —~pur %%%%%
% completo~—.
%
% Quanto al contenuto  in senso proprio,
% la classe prevede che  il luogo adatto
% siano i seguenti comandi.  Il loro uso
% non dev'essere per forza vincolato %%%
% dalle funzioni esemplificate in questa
% documentazione,  perché questi comandi
% possono prestarsi, anzitutto, a %%%%%%
% funzioni analoghe a quelle previste.
%
% \DescribeMacro{\cvdescription} %%%
% Il comando \textsf{description} è solo
% accessorio ai successivi, perché serve
% a una descrizione più dettagliata %%%%
% d'informazioni rese d'altri comandi. %
% \`{E}, dunque, un comando trasversale.
% Ha un solo argomento, che la classe %%
% stampa in dimensione \texttt{footnote}
% e col margine sinistro allungato  pari
% all'indentatura del titolo della %%%%%
% sezione. Il testo è giustificato.  Non
% serve per le informazioni di base,  ma
% a darne una precisazione,  che a volte
% non è strettamente indispensabile alla
% validità dell'informazione stessa,  ma
% che serve, per esempio, a darle un %%%
% contesto.
%
% I comandi  per le informazioni di base
% sono essenzialmente analoghi con %%%%%
% riferimento sia all'\textit{input}, %%
% che, in effetti , all'\textit{output}.
% Infatti, sono sovrapponibili, tutto %%
% sommato,  perché riconducibili al %%%%
% comando \textsf{cvitem}. L'esposizione
% in questa documentazione, tuttavia, %%
% segue l'ordine classico in cui un %%%%
% \textit{cv} presenta le informazioni.
%
% \DescribeMacro{\cveducation}
% Le informazioni sui titoli conseguiti,
% o, comunque, sulle attività formative,
% si compongono essenzialmente di tre %%
% dati:  la data del conseguimento (o in
% cui l'evento formativo si tiene), il %
% nome del titolo o dell'evento e un %%%
% sottotitolo o l'indicazione della %%%%
% natura del titolo o dell'evento.  %%%%
% Questi tre dati sono, in ordine, i tre
% argomenti del comando denominato \textsf
% {cveducation}.  La classe stampa
% \begin{itemize}
%     \item la data sbandierata a destra
%         e posta come in una colonna, %
%         di larghezza  pari alla misura
%         dell'indentazione del titolo %
%         delle sezioni, e che è scritta
%         in tondo;
%     \item il nome del titolo o dell'evento,
%         in neretto, e
%     \item il sottotitolo,  o la natura
%         del titolo o dell'evento, in %
%         tondo, con virgola a precedere.
% \end{itemize}
% Questa struttura può andare bene anche
% per esperienze lavorative,  associando
% agli argomenti un diverso significato,
% per analogia.  Così, possono prestarsi
% a questa struttura i tirocini.  In %%%
% questa struttura, il comando \textsf%%
% {cvdescription}  è un'opzione per dare
% informazioni sull'istituto frequentato
% o sul luogo in cui si svolge, o s'è %%
% svolto, il tirocinio, o, quindi, sul %
% datore di lavoro.  In astratto, niente
% impedisce, però, di dire istituto, %%%
% impresa o datore di lavoro nel terzo %
% argomento del comando \textsf{cveducation}.
%
% La struttura di \textsf{cveducation} è
% abbastanza ingabbiata, in ogni caso. %
% Questo vale, soprattutto, dal punto di
% vista della formattazione.
%
% \DescribeMacro{\cvitem}  \`{E}
% decisamente più libera, infatti, la %%
% struttura del comando \textsf{item}, %
% che ha solo due argomenti:
% \begin{itemize}
%     \item come per il comando  \textsf
%         {cveducation}, il primo  viene
%         scritto in tondo,  sbandierato
%         a destra e posto come in una %
%         colonna, pari alla misura %%%%
%         dell'indentazione del titolo %
%         delle sezioni, prestandosi, in
%         ogni caso, a un contenuto  che
%         non dev'essere necessariamente
%         una data, ma che è bene sia, %
%         comunque, un testo brevissimo;
%     \item il secondo argomento è in %%
%         tondo semplice,  ben potendosi
%         prestare a qualsiasi contenuto
%         e a qualsiasi formattazione.
% \end{itemize}
% Dunque,  nei fatti, questa struttura è
% una generalizzazione del comando \textsf
% {cveducation}. Si presta, per esempio,
% all'indicazione di competenze,  perché
% il primo argomento può indicare l'area
% di competenza e il secondo elencare le
% specifiche competenze  qui rientranti.
% In particolare, il secondo argomento %
% assorbe in uno solo il secondo e il %%
% terzo argomento  del  comando  \textsf
% {cveducation}.
%
% Il comando \textsf{description} ben %%
% può accedere anche a \textsf{cvitem}.
%
% \DescribeMacro{\cvcontribution}
% Invece, \textsf{cvcontribution} ha tre
% argomenti, di nuovo, e nell'\textit%%%
% {input} si distingue dal comando \textsf
% {cveducation} in quanto il secondo è %
% stampato in italico  e non in neretto.
% Questo ha lo scopo precipuo di dare %%
% massima esemplificazione all'elencazione
% d'eventuali contributi o pubblicazioni
% in riviste e simili, posto che il loro
% titolo, secondo le maggiori prassi %%%
% citazionali —~almeno, in Italia~—, va,
% appunto, in italico.  A parte questa %
% distinzione di formattazione,  dunque,
% \textsf {cvcontribution} ha sempre tre
% argomenti, che, tipicamente, sono
% \begin{itemize}
%     \item la data sbandierata a destra
%         e posta come in una colonna, %
%         di larghezza  pari alla misura
%         dell'indentazione del titolo %
%         delle sezioni, e che è scritta
%         in tondo;
%     \item il nome del titolo, in %%%%%
%         italico, e
%     \item ulteriori informazioni,  che
%         vengono precedute dalla %%%%%%
%         virgola.
% \end{itemize}
% Il comando \textsf{description} ben %%
% può accedere qui:  per esempio, se è %
% indispensabile, può essere lo spazio %
% per un breve \textit{abstract}.  Se %%
% un sommario non è necessario, però, il
% consiglio è di non dare alcuna \textsf
% {description} al contributo indicato.
%
%
%
% \iffalse
%<*example>
% \fi
\begin{lstlisting}[caption={Schema di
    \textit{cv}},style=inputexample,
    showspaces=false,
    label={code:cvoutline}]
\documentclass[public]{curriculum}
    \name{Nome}
    \surname{Cognome}
    \position{Laureata}
    \birthday{01.01.2000}
    \gender{donna}

    \addresslineone{via delle Lumache, 15}
    \addresslinetwo{00100 -- Roma, Italia}
    \email{username@domain.com}
    \pec{username@pecdomain.com}
    \phone{+39 06 0000 0000}
    \fax{+39 06 0000 0000}
    \mobile{+39 300 000 0000}
    \github{username}
    \linkedin{username}

    \portrait{mishlanova-unsplash.jpg}

    \beforedate{ult.~agg\@.}
    \place{Roma}

\begin{document}
    \maketitle
    \section{Sezione [titolo]}
    \subsection{Sottosezione [titolo]}
        \cveducation{2018}
            {Grado conseguito}
            {specificazione}
        \cvdescription{Questo è il luogo per %
                scrivere la scuola o dare %%%%
                altre informazioni.}
        \cvitem{anche parole}
            {Elemento o testo}
        \cvcontribution{2018}
            {Titolo contributo}
            {informazioni}

    \begin{signature}
        {Ambiente di firma [titolo]}
        Testo.
    \end{signature}
\end{document}
\end{lstlisting}
% \iffalse
%</example>
% \fi
%
% \begin{figure}
%     \centering
%     \includegraphics[height=.9\textheight]
%         {cvoutline.pdf}
%     \label{fig:cvoutline}
%     \caption{Anteprima del codice \ref
%         {code:cvoutline}}
% \end{figure}
%
%
%
% \DescribeEnv{signature} Infine, c'è %%
% l'ambiente \textsf{signature}. \`{E} %
% il luogo perfetto per inserire una %%%
% dichiarazione di veridicità dei dati e
% delle informazioni  che il \textit{cv}
% rende, nonché un'autorizzazione al %%%
% loro trattamento. Questo ambiente ha %
% un suo argomento, che corrisponde al %
% titoletto che precede la dichiarazione
% e l'autorizzazione, o, comunque, il %%
% testo che lo scrivente vuole inserire.
% Questo titoletto è di pari livello, in
% astratto, a quello delle sezioni di %%
% primo livello, ma non ha indentazione.
% Il testo dell'ambiente, invece, se %%%
% dato, è stampato in \texttt{small} e %
% sta giustificato.
%
% L'ambiente \textsf{signature} termine,
% nell'\textit{output} con l'indicazione
% della data (eventualmente, anche del %
% luogo) di scrittura del \textit{cv} e,
% poi, con un'area per la sottoscrizione.
%
% \subsection{La lettera di presentazione}
% \DescribeEnv{introduction}
% La classe, per ultimo, prevede un'area
% per una lettera di presentazione. %%%%
% L'utente, dunque, può precedere il suo
% \textit{curriculum vitae et studiorum}
% con una lettera introduttiva,  con suo
% oggetto e espressamente indirizzata %%
% a uno specifico destinatario.
%
% L'ambiente ha quattro argomenti %%%%%%
% obbligatori:
% \begin{enumerate}
%     \item il nome del destinatario, %%
%         stampato in neretto,
%     \item l'indirizzo del destinatario
%         completo, per cui l'utente può
%         anche andare a capo, col %%%%%
%         comando \texttt{\textbackslash
%         par}, e che la classe stampa %
%         in tondo,
%     \item l'oggetto, che la classe %%%
%         stampa in neretto, e, infine,
%     \item la formula introduttiva.
% \end{enumerate}
% I primi due argomenti compongono il %%
% destinatario.
%
% La classe stampa il destinatario in %%
% uno spazio apposito nella parte destra
% della gabbia del testo.  Il testo è, %
% comunque, sbandierato a sinistra.
%
% Anche l'oggetto è stampato sbandierato
% a sinistra,  ma occupa l'intera gabbia
% del testo.
%
% Il corpo della lettera, comprensivo %%
% della formula introduttiva, invece, %%
% è giustificato e sta in una gabbia %%%
% dai margini ristretti.  Questo per %%%
% facilitarne la lettura, dato che non è
% un mero insieme di dati e informazioni
% in elenco ordinato, ma, appunto, un %%
% testo.
%
% Quanto alla chiusura della lettera, la
% classe non prevede un apposito comando
% per la formula di saluto finale, anche
% se l'utente può, certo, apporla di suo
% proposito.
%
% Invece, la classe prevede, a seguire %
% l'ambiente \textsf{introduction}, lo %
% spazio per la sottoscrizione.  \`{E} %
% lo stesso spazio  creato dall'ambiente
% \textsf{signature}. Però, a differenza
% di  quest'ultimo,  l'ambiente  \textsf
% {introduction} non ristampa luogo,  se
% l'utente l'ha specificato, e data di %
% firma.  D'altra parte, l'utente può %%
% può valutare di non usare l'ambiente %
% \textsf{signature} a chiusura del %%%%
% \textit{cv}, per limitarsi a firmare %
% la lettera di presentazione stessa. In
% quest'ultimo caso, semmai, il problema
% è spostare eventuali dichiarazioni e %
% le autorizzazioni al trattamento dei %
% dati a fine lettera introduttiva.
%
% \DescribeEnv{enclosed}  Per ultimo, la
% classe mette a disposizione l'apposito
% ambiente per l'elenco d'eventuali %%%%
% allegati alla lettera introduttiva. %%
% C'è un ambiente obbligatorio e serve %
% a esplicitare che l'elenco associato %
% è relativo agli allegati. Il contenuto
% dell'ambiente è un elenco: pertanto, %
% le voci devono essere precedute dal %%
% comando \texttt{\textbackslash item}.
%
% Terminata la lettera introduttiva, %%%
% è, dunque, possibile incominciare il %
% vero e proprio \textit{cv}. A questo %
% fine,  è sufficiente inserire di nuovo
% \texttt{\textbackslash maketitle}, %%%
% comando che interrompe la pagina e %%%
% stampa nuovamente l'intestazione.  Nel
% caso, i margini della gabbia del testo
% si ripristinano. La prima pagina del %
% \textit{cv} è, dunque, sempre priva di
% testatine e piedini. Inoltre, prosegue
% la numerazione delle pagine, senza %%%
% azzeramento, tra la lettera e il \textit
% {cv}.   Inoltre, tra lettera e \textit
% {cv}, niente cambia con riferimento %%
% alle opzioni di riservatezza  sui dati
% personali.
%
% Nulla impedisce d'usare la classe, con
% l'ambiente \texttt{introduction}, %%%%
% eclusivamente per scrivere lettere %%%
% motivazionali o di presentazione.
%
%
%
% \iffalse
%<*example>
% \fi
\begin{lstlisting}[caption={Schema di
        lettera di presentazione},
        style=inputexample,
        showspaces=false,
        label={code:introduction}]
\documentclass{curriculum}
    \name{Nome}
    \surname{Cognome}
    \position{Laureata}
    \birthday{01.01.2000}
    \gender{donna}

    \addresslineone{via delle Lumache, 15}
    \addresslinetwo{00100 -- Roma, Italia}
    \email{username@domain.com}
    \pec{username@pecdomain.com}
    \phone{+39 06 0000 0000}
    \fax{+39 06 0000 0000}
    \mobile{+39 300 000 0000}
    \github{username}
    \linkedin{username}

    \portrait{mishlanova-unsplash.jpg}

    \place{Roma}

\usepackage{polyglossia}
    \setmainlanguage{italian}
    \SetLanguageKeys{italian}{indentfirst=false}

\begin{document}
    \maketitle
    \begin{introduction}
            {Società Alfa s.r.l\@.}
            {Area \textit{recruiter}\par
                via delle Lumache, 19\par
                00100 – Roma, Italia\par
                \textbf{\MakeUppercase{pec}}
                \texttt{alfasrl@fakepec.it}}
            {Lettera motivazionale
                e invio \textit{cv}
                (\textit{infra})}
            {Gentili,}
        ho visto il Vs.~avviso di selezione.
        Dinnanzi all'opportunità di lavorare
        presso la Vs.~società nutro sfrenato
        entusiamo.

        In allegato unico trovate il \textit
        {curriculum vitae et studiorum}, nel
        quale riassumo quelle esperienze e %
        quelle competenze  che mi rendono un
        profilo assai adatto alla Vs.~ricer%
        ca.

        Gradirei molto tenere un incontro %%
        conoscitivo.

        Resto in attesa del Vs.~riscontro.

        Distinti saluti,
    \end{introduction}
    \begin{enclosed}{Allegati}
        \item \textit{curriculum vitae} di %
            seguito e, a parte,
        \item modulo domanda compilato,
        \item copia digitale del documento %
            d'identità,
        \item lettera di referenza dell'av%%
            v.a~Sabrina Bianchi,
        \item lettera di referenza dell'av%%
            v.~Mario Rossi.
    \end{enclosed}
\end{document}
\end{lstlisting}
% \iffalse
%</example>
% \fi
%
% \begin{figure}
%     \centering
%     \includegraphics[height=.9\textheight]
%         {introduction.pdf}
%     \label{fig:introduction}
%     \caption{Anteprima del codice \ref
%         {code:introduction}}
% \end{figure}
%
%
%
% \StopEventually{}
% \iffalse
%<*class>
% \fi
% \section{Il codice}
% \subsection{Nuovi comandi}
% Il codice esordisce con l'impostazione
% dei comandi relativi ai dati personali
% e con la previsione di due gruppi, che
% raccolgono questi stessi dati.
%    \begin{macrocode}
\DeclareRobustCommand\name[1]
    {\gdef\@name{#1}}
\DeclareRobustCommand\surname[1]
    {\gdef\@surname{#1}}

\DeclareRobustCommand\matter[1]
    {\gdef\@matter{#1}}
    \global\let\@matter\@empty

\DeclareRobustCommand\beforedate[1]
    {\gdef\@beforedate{#1}}
    \global\let\@beforedate\@empty
\DeclareRobustCommand\place[1]
    {\gdef\@place{#1}}
    \global\let\@place\@empty

\DeclareRobustCommand\position[1]
    {\gdef\@position{#1}}
    \global\let\@position\@empty
\DeclareRobustCommand\birthday[1]
    {\gdef\@birthday{#1}}
    \global\let\@birthday\@empty
\DeclareRobustCommand\gender[1]
    {\gdef\@gender{#1}}
    \global\let\@gender\@empty

\DeclareRobustCommand\addresslineone[1]
    {\gdef\@addresslineone{#1}}
    \global\let\@addresslineone\@empty
\DeclareRobustCommand\addresslinetwo[1]
    {\gdef\@addresslinetwo{#1}}
    \global\let\@addresslinetwo\@empty

\DeclareRobustCommand\email[1]
    {\gdef\@email{#1}}
    \global\let\@email\@empty

\DeclareRobustCommand\pec[1]
    {\gdef\@pec{#1}}
    \global\let\@pec\@empty

\DeclareRobustCommand\phone[1]
    {\gdef\@phone{#1}}
    \global\let\@phone\@empty
\DeclareRobustCommand\fax[1]
    {\gdef\@fax{#1}}
    \global\let\@fax\@empty

\DeclareRobustCommand\mobile[1]
    {\gdef\@mobile{#1}}
    \global\let\@mobile\@empty

\DeclareRobustCommand\linkedin[1]
    {\gdef\@linkedin{#1}}
    \global\let\@linkedin\@empty
\DeclareRobustCommand\github[1]
    {\gdef\@github{#1}}
    \global\let\@github\@empty

\DeclareRobustCommand\portrait[1]
    {\gdef\@portrait{#1}}
    \global\let\@portrait\@empty

%    \end{macrocode}
% I gruppi sono quelli attraverso cui la
% classe opera  per attuare le eventuali
% opzioni di riservatezza,  che l'utente
% voglia, eventualmente, attivare, con %
% \texttt{public} o \texttt{privacy}.
%
% \subsection{Opzioni della classe}
% Di \textit{default},  la classe stampa
% tutti i dati che l'utente fornisce,  a
% meno che l'utente non indichi opzioni,
% che dispongano la soppressione d'uno o
% più dati personali.  Tra le opzioni, %
% è di \textit{default} la stampa del %%
% documento a colori, anche se l'opzione
% \texttt{blackandwhite} consente di %%%
% stampare il documento in bianco e nero
% e grigio, salvo solo l'eccezione della
% data e dell'oggetto del documento e, %
% ovviamente, dell'eventuale fotografia.
%    \begin{macrocode}
\DeclareOption{twocolumn}{\OptionNotUsed}

\DeclareOption{cvcolors}{\DeclareRobustCommand{\cvcolors}{%
        \definecolor{cvblue}{rgb}{0.25,0.25,0.50}
        \definecolor{cvgreen}{rgb}{0.25,0.50,0.25}
        \definecolor{cvprename}{rgb}{0.35,0.60,0.35}
        \definecolor{cvsurname}{rgb}{0.25,0.50,0.25}
        \color{cvblue}}}
\DeclareOption{blackandwhite}{\renewcommand{\cvcolors}{%
        \definecolor{cvblue}{gray}{0.25}
        \definecolor{cvgreen}{gray}{0.25}
        \definecolor{cvprename}{gray}{0.25}
        \definecolor{cvsurname}{gray}{0.00}
        \color{black}}}

\DeclareOption{nobirthdaynogender}{\renewcommand{\birthday}[1]{\gdef\@birthday{\@empty}%
        \let\@birthday\@empty}
    \renewcommand{\gender}[1]{\gdef\@gender{\@empty}%
        \let\@gender\@empty}}
\DeclareOption{noaddress}{\renewcommand{\addresslineone}[1]{\gdef\@addresslineone{\@empty}%
        \let\@addresslineone\@empty}
    \renewcommand{\addresslinetwo}[1]{\gdef\@addresslinetwo{\@empty}%
        \let\@addresslinetwo\@empty}
    \renewcommand{\pec}[1]{\gdef\@pec{\@empty}%
        \let\@pec\@empty}
    \renewcommand{\phone}[1]{\gdef\@phone{\@empty}%
        \let\@phone\@empty}
    \renewcommand{\fax}[1]{\gdef\@fax{\@empty}%
        \let\@fax\@empty}}
\DeclareOption{letterhead}{\ExecuteOptions{nobirthdaynogender}
    \renewcommand{\portrait}[1]{\gdef\@portrait{\@empty}%
        \let\@portrait\@empty}}
\DeclareOption{privacy}{\ExecuteOptions{nobirthdaynogender,public}
    \renewcommand{\pec}[1]{\gdef\@pec{\@empty}%
        \let\@pec\@empty}}
\DeclareOption{public}{\ExecuteOptions{noaddress}
    \renewcommand{\mobile}[1]{\gdef\@mobile{\@empty}%
        \let\@mobile\@empty}}

\DeclareOption*{\PassOptionsToClass{\CurrentOption}{article}}

%    \end{macrocode}
% L'utente può inserire opzioni che la %
% classe  non  prevede,  tranne  \texttt
% {twocolumn}: \textsf{curriculum}, %%%%
% espressamente, la esclude. Quanto alle
% altre eventuali opzioni, sono possibili,
% ma sconsigliate, viste le ripercussioni.
%
% Dunque, \textsf{curriculum} esegue, di
% \textit{default}, \texttt{cvcolors}  e
% imposta il documento in formato %%%%%%
% \textsc{iso} A4, con testo corrente di
% di dimensione 12pt.
%    \begin{macrocode}
\ExecuteOptions{cvcolors}
\ProcessOptions\relax
\LoadClass[12pt,a4paper]{article}

%    \end{macrocode}
%
% \subsection{Caricamento dei pacchetti fondamentali}
% Fatte queste impostazioni fondamentali,
% la classe carica i pacchetti, continua
% a personalizzare il \textit{layout} %%
% del documento e specifica l'essenziale
% pacchetto \texttt{hyperref}.
%    \begin{macrocode}
\RequirePackage{array}
\RequirePackage{changepage}
\RequirePackage[ddmmyyyy]{datetime}
    \newcommand{\defaulttoday}
        {\twodigit{\day}.\twodigit{\month}.\twodigit{\year}}
\RequirePackage{fancyhdr}
    \setlength{\hoffset}{8pt}
    \setlength{\oddsidemargin}{0pt}
    \setlength{\textwidth}{437pt}
    \setlength{\marginparsep}{0pt}
    \setlength{\marginparwidth}{0pt}
    \setlength{\voffset}{-42pt}
    \setlength{\topmargin}{0pt}
    \setlength{\headheight}{25pt}
    \setlength{\headsep}{16pt}
    \setlength{\textheight}{668pt}
    \AtEndDocument{\label{lastpage}}
    \fancyhf{}
    \fancyhf[HC]{\color{gray}{\footnotesize%
        \begin{spacing}{.8}
            \textbf{\@name\ \@surname} \ifx\@position\@empty\else
                ---\ \textit{\bfseries\@position}\fi\par
            \ifx\@matter\@empty\else
                \@matter\fi
        \end{spacing}}}
    \fancyhf[FC]{\color{gray}{\footnotesize%
        \begin{spacing}{.8}
            \MakeUppercase{\bfseries\@name\ \@surname}\par
            \ifx\@addresslineone\@empty%
                \else
                \@addresslineone%
                \ifx\@addresslinetwo\@empty%
                    \else
                    \enspace\@addresslinetwo\fi\par\fi
            \ifx\@phone\@empty%
                \ifx\@fax\@empty%
                    \ifx\@mobile\@empty%
                        \else\faMobile\ \@mobile\par\fi
                    \else\faFax\ \@fax\ $\bullet$\ \faMobile\ \@phone\par\fi
                \else\faPhoneSquare\ \@phone\ifx\@fax\@empty%
                    \ifx\@mobile\@empty\par
                    \else\ $\bullet$\ \faMobile\ \@mobile\par\fi
                    \else\ $\bullet$\ \faFax\ \@fax%
                    \ifx\@mobile\@empty\par
                        \else\ $\bullet$\ \faMobile\ \@mobile\par\fi\fi\fi
            \ifx\@email\@empty%
                \ifx\@pec\@empty\else\href{mailto:\@pec}{\faCertificate\ \@pec}\par\fi
                \else
                \href{mailto:\@email}{\faEnvelopeO\ \@email}%
                \ifx\@pec\@empty\par\else\ $\bullet$\ \href{mailto:\@pec}{\faCertificate\ \@pec}\fi\par\fi
            \ifx\@linkedin\@empty
                \ifx\@github\@empty
                \else
                \href{https://github.com/\@github}
                    {\faGithub\ \@github}\par\fi\else
                \href{https://linkedin.com/in/\@linkedin}
                    {\faLinkedinSquare\ \@linkedin}
                \ifx\@github\@empty\else$\bullet$\ \href{https://github.com/\@github}
                    {\faGithub\ \@github}\fi\fi%
        \end{spacing}}}
    \fancyhf[FR]{\color{darkgray}\em\footnotesize%
        \begin{spacing}{.8}\hspace{1pt}\par%
            \thepage/\pageref{lastpage}
        \end{spacing}}
    \renewcommand{\headrulewidth}{0pt}
    \renewcommand{\footrulewidth}{0pt}
    \renewcommand{\headruleskip}{15mm}
    \renewcommand{\footruleskip}{15pt}
\RequirePackage{fontawesome}
\RequirePackage{graphicx}
\RequirePackage{ifthen}
\RequirePackage{paralist}
    \RequirePackage{enumitem}
    \setlist[itemize]{label={-},noitemsep,topsep=0pt}
    \setlist[enumerate]{noitemsep,topsep=0pt}
    \setlist[description]{noitemsep,topsep=0pt}
\RequirePackage{ragged2e}
\RequirePackage{setspace}
\RequirePackage{tabularx}
\RequirePackage{xcolor}
    \cvcolors
\RequirePackage{xparse}

\pagestyle{fancy}
\raggedbottom

\RequirePackage{hyperref}

%    \end{macrocode}
%
% \subsection{Personalizzazione di sezioni e titoli}
% A questo punto, il codice è pronto per
% descrivere l'intestazione del  \textit
% {cv}.  A questo proposito, il codice %
% sfrutta il comune comando \texttt{\textbackslash
% maketitle}.
%    \begin{macrocode}
\renewcommand{\maketitle}{\clearpage%
    \thispagestyle{empty}%
    \noindent\begingroup%
    \begin{minipage}[m][125pt][t]{.40\textwidth}
        \begingroup%
        \Large%
        \textcolor{cvprename}{\bfseries\@name}
        \textcolor{cvsurname}{\mbox{\bfseries\@surname}}\par
        \large
        \textit{\@position}\par\medskip
        \small
        \ifx\@birthday\@empty%
            \else
            \faBirthdayCake\ \@birthday\par\fi
        \ifthenelse{\equal{\@gender}{F}
            \OR
            \equal{\@gender}{f}
            \OR
            \equal{\@gender}{Female}
            \OR
            \equal{\@gender}{Female}
            \OR
            \equal{\@gender}{Woman}
            \OR
            \equal{\@gender}{woman}
            \OR
            \equal{\@gender}{Femmina}
            \OR
            \equal{\@gender}{femmina}
            \OR
            \equal{\@gender}{Donna}
            \OR
            \equal{\@gender}{donna}}
            {\faVenus\ \@gender\par}
            {\ifthenelse{\equal{\@gender}{M}
                \OR
                \equal{\@gender}{m}
                \OR
                \equal{\@gender}{Male}
                \OR
                \equal{\@gender}{male}
                \OR
                \equal{\@gender}{Man}
                \OR
                \equal{\@gender}{man}
                \OR
                \equal{\@gender}{Maschio}
                \OR
                \equal{\@gender}{maschio}
                \OR
                \equal{\@gender}{Uomo}
                \OR
                \equal{\@gender}{uomo}}
                {\faMars\ \@gender\par}
                {\if\@gender\@empty\else\@gender\fi}}\par
        \bigskip
        \normalsize
        \ifx\@matter\@empty%
            \else
            {\bfseries\@matter}\par\fi
        \ifx\@beforedate\@empty%
            \else
            \@beforedate~\fi
        \ifx\@place\@empty%
            \else
            \@place\ \fi\defaulttoday\par
        \endgroup
    \end{minipage}
    \hspace{.02pt}
    \begin{minipage}[m][125pt][t]{.35\textwidth}
        \begingroup
        \footnotesize
        \ifx\@addresslineone\@empty%
            \else
            \@addresslineone\par\fi
        \ifx\@addresslinetwo\@empty%
            \else
            \@addresslinetwo\par\fi
        \ifx\@email\@empty%
            \else
            \href{mailto:\@email}{\faEnvelopeO\ \@email}\par\fi
        \ifx\@pec\@empty%
            \else
            \href{mailto:\@pec}{\faCertificate\ \@pec}\par\fi
        \ifx\@phone\@empty%
            \else
            \faPhone\ \@phone\par\fi
        \ifx\@fax\@empty%
            \else
            \faFax\ \@fax\par\fi
        \ifx\@mobile\@empty%
            \else
            \faMobile\ \@mobile\par\fi
        \ifx\@linkedin\empty%
            \else
            \href{https://linkedin.com/in/\@linkedin}
                {\faLinkedinSquare\ \@linkedin}\par\fi
        \ifx\@github\empty%
            \else
            \href{https://github.com/\@github}
                {\faGithub\ \@github}\par\fi
        \endgroup
    \end{minipage}
    \ifx\@portrait\@empty%
        \else
        \hfill\begin{minipage}[m][125pt][t]{.23\textwidth}
            \fbox{\includegraphics
                [width=.75\textwidth]
                {\@portrait}}
        \end{minipage}\fi
    \endgroup\bigskip}

%    \end{macrocode}
%
% Di seguito, personalizza il formato %%
% dell'intestazione delle sezioni.
%    \begin{macrocode}
\setcounter{secnumdepth}{0}
\renewcommand{\section}{%
    \@startsection
    {section}{1}{0pt}{8pt}%
    {4pt}{\noindent\large\scshape\color{cvgreen}\rule[2pt]{76.475pt}{4pt}\hspace{4.37pt}}}
\renewcommand{\subsection}{%
    \@startsection
    {subsection}{1}{0pt}{4pt}%
    {4pt}{\noindent\large\sffamily\color{cvgreen}}}

%    \end{macrocode}
%
%
% Infine, la classe crea comandi e %%%%%
% ambienti.
%
% Anzitutto, la classe crea i comandi %%
% per il contenuto in senso stretto  del
% \textit{cv}.
%    \begin{macrocode}
\newcommand{\cveducation}[3]{\vspace*{4pt}\noindent%
    \begingroup%
    \setlength{\tabcolsep}{0pt}%
    \begin{tabular}
        {>{\raggedleft\raggedbottom}p{.175\textwidth}
            @{\hskip .01\textwidth}
            >{\raggedright\raggedbottom}p{.815\textwidth}}
        #1 & \textbf{#2}, #3
    \end{tabular}
    \endgroup\par}
\newcommand{\cvdescription}[1]{\vspace*{-2pt}\noindent%
    \begingroup%
    \setlength{\tabcolsep}{0pt}%
    \begin{tabular}
        {>{\raggedleft\raggedbottom}p{.185\textwidth}
            >{\raggedright\raggedbottom}p{.815\textwidth}}
        & \footnotesize\setstretch{.8}#1
    \end{tabular}
    \endgroup\par}
\newcommand{\cvitem}[2]{\vspace*{4pt}\noindent%
    \begingroup%
    \setlength{\tabcolsep}{0pt}%
    \begin{tabular}
        {>{\raggedleft\raggedbottom}p{.175\textwidth}
            @{\hskip .01\textwidth}
            >{\raggedright\raggedbottom}p{.815\textwidth}}
        #1 & #2
    \end{tabular}
    \endgroup\par}
\newcommand{\cvcontribution}[3]{\vspace*{4pt}\noindent%
    \begingroup%
    \setlength{\tabcolsep}{0pt}%
    \begin{tabular}
        {>{\raggedleft\raggedbottom}p{.175\textwidth}
            @{\hskip .01\textwidth}
            >{\raggedright\raggedbottom}p{.815\textwidth}}
        #1 & \textit{#2}, #3
    \end{tabular}
    \endgroup\par}

%    \end{macrocode}
%
% Poi, la classe crea un ambiente per la
% sottoscrizione del \textit{cv}, in cui
% l'utente possa pure dichiarare la %%%%
% veridicità di dati e informazioni e %%
% autorizzare il relativo trattamento.
%    \begin{macrocode}
\newenvironment{signature}[1]{\begingroup%
    \renewcommand{\section}{%
        \@startsection%
        {section}{1}{0pt}{8pt}%
        {4pt}{\noindent\large\scshape\color{cvgreen}}}%
    \section*{#1}
    \setlength\parindent{0pt}%
    \small\setstretch{.8}}
    {\par\nopagebreak\smallskip\ifx\@place\@empty%
        \defaulttoday\else
        \@place\ \defaulttoday\fi.\par\nopagebreak\vskip 1cm
        \hfill\makebox[.35\textwidth][c]
        {\rule[0pt]{.35\textwidth}{0.15pt}}%
        \hspace*{.1\textwidth}\par\nopagebreak
        \hfill\makebox[.35\textwidth][c]
        {\footnotesize\textsc{\@name\ \@surname}}%
        \hspace*{.1\textwidth}\par
        \endgroup}

%    \end{macrocode}
%
% Infine, la classe crea l'ambiente %%%%
% \textsf{introduction},  per la lettera
% di presentazione.
%    \begin{macrocode}
\newenvironment{introduction}[4]{%
    \begin{adjustwidth}{.65\textwidth}{0pt}%
        \setlength{\parindent}{0pt}%
        \raggedright%
        {\bfseries#1}\par
        #2
    \end{adjustwidth}%
    \begin{flushleft}
        {\bfseries#3}
    \end{flushleft}%
    \begingroup%
    \begin{adjustwidth}{45pt}{45pt}%
        \begingroup%
        \setlength{\parskip}{12pt}%
        \noindent#4\par}
    {\endgroup\vskip 1cm
        \hfill\makebox[.35\textwidth][c]
        {\rule[0pt]{.35\textwidth}{0.15pt}}%
        \hspace*{.1\textwidth}\par\nopagebreak
        \hfill\makebox[.35\textwidth][c]
        {\footnotesize\textsc{\@name\ \@surname}}%
        \hspace*{.1\textwidth}\par
    \end{adjustwidth}%
    \endgroup}

%    \end{macrocode}
%
% A corredo della lettera di motivazione
% o di presentazione, la classe crea %%%
% l'ambiente \textsf{enclosed}, per fare
% un elenco degli allegati.
%    \begin{macrocode}
\newenvironment{enclosed}[1]
    {\begin{minipage}[t]{.1725\textwidth}%
        \rule[0pt]{1cm}{0pt}\par\RaggedLeft{\emph{#1}}
    \end{minipage}\hspace{.005\textwidth}
    \begin{minipage}[t]{.8225\textwidth}%
        \rule[0pt]{5cm}{0pt}\begin{enumerate}}
    {\end{enumerate}\end{minipage}}
%    \end{macrocode}
% \iffalse
%</class>
% \fi
%
%
%
% \iffalse
%<*data>
\Title{The curriculum class}
\Author{Francesco Contini}
\Language{it-IT}
\Date{2025-08-21}
\Creator{LuaLaTeX}
\Subject{Curriculum class}
\Keywords{cv\sep class\sep latex}
%</data>
% \fi
%
%
% \iffalse
%<*bib>
    @online{contini2025,
        author={Contini, Francesco},gender={sm},
        title={\texttt{curriculum.cls}},
        organization={GitHub},
        url={https://github.com/francesco-contini/curriculumclass},
        date={2025-08-21}}
    @manual{danaux2024,
        author={Danaux, Xavier},gender={sm},
        title={\texttt{moderncv}\ User Guide},
        editor={Adlkofer, Daniel and Lachnit, Stephan},
        version={2.4.1},
        publisher={\textsc{ctan}},
        url={https://ctan.org/pkg/moderncv},
        date={2024-07-18}}
    @manual{ghersi2012,
        author={Ghersi, Andrea},gender={sm},
        title={MyCV},
        version={1.16},
        publisher={\textsc{ctan}},
        url={https://ctan.org/pkg/mycv},
        date={2012-05-20}}
    @manual{mazzamuto2020,
        author={Mazzamuto, Giacomo},gender={sm},
        title={Documentation of the \LaTeX\ class
            \texttt{europasscv.cls}},
        publisher={\textsc{ctan}},
        url={https://ctan.org/pkg/europasscv},
        date={2020-10-31}}
    @online{maier2001,
        author={Maier, Philipp},gender={sm},
        title={\texttt{cv.sty}},
        subtitle={A package for creating a curriculum
            vitae},
        publisher={\textsc{ctan}},
        url={https://ctan.org/pkg/cv},
        date={2001-03-01}}
    @online{mishlanova2021,
        author={Mishlanova, Elena},gender={sf},
        title={Donna in camicia},
        type={immagine},
        organization={unsplash},
        url={https://unsplash.com/it/foto/donna-in-camicia-rosa-e-pantaloni-grigi-XGXn9JiSEvM},
        urldate={2025-08-25},
        date={2021-04-21},
        addendum={ritaglio}}
    @online{prasad2023,
        author={Prasad, Sumukh},gender={sm},
        title={\texttt{curriculum-vitae.cls}},
        organization={GitHub},
        version={2.0},
        url={https://github.com/SumukhPrasad/latex-curriculum-vitae},
        date={2025-07-09}}
    @online{scrivena2018,
        author={Scrivena, Zach},gender={sm},
        title={\texttt{simpleresumecv.cls}},
        version={3.0},
        publisher={GitHub},
        url={https://github.com/zachscrivena/simple-resume-cv},
        date={2018-12-02}}
    @manual{verna2010,
        author={Verna, Didier},gender={sm},
        title={Curve – a \LaTeXe\ class package
            for making \textbf{Cur}ricula \textbf
            {V}ita\textbf{e}},
        version={1.16},
        publisher={\textsc{ctan}},
        url={https://ctan.org/pkg/curve},
        date={2010-12-14}}
%</bib>
% \fi
%
%
%
% \Finale
% \endinput
% \fi